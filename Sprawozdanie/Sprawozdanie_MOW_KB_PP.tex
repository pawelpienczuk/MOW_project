%%%%%%%%%%%%%%%%%%%%%%%%%%%%%%%%%%%%%%%%%%%%%%%%%%%%%%%%%%%%%  
%
% Sprawozdanie MOW
% TEMAT PROJEKTU: 
% AUTOR: 	Krzysztof Belewicz
%			Paweł Pieńczuk
% 23.01.2020
%
%%%%%%%%%%%%%%%%%%%%%%%%%%%%%%%%%%%%%%%%%%%%%%%%%%%%%%%%%%%%%

\documentclass[a4paper,11pt,twoside]{mwrep}  %bylo report
\usepackage[utf8]{inputenc} %ISO-8859-2 
\usepackage[UKenglish,polish]{babel} 
\usepackage[T1]{fontenc} 
\usepackage{mathptmx}
\usepackage[scaled=.90]{helvet}
\usepackage{courier}
\usepackage{gensymb} %\degree
%\usepackage[pdftex]{graphicx} 
%\usepackage{pdfpages} 
%\usepackage{wrapfig}
%\usepackage{hyperref} 
%\usepackage{gensymb}
% tablice
%\usepackage{ctable}
%\usepackage{threeparttablex}
%\usepackage{booktabs}
% grafiki
%\usepackage{graphicx}
%\usepackage{subfig}
%\usepackage{float}
%% matematyka i greckie literki w jednostkach
%\usepackage{amsmath}
%\usepackage{textgreek}
%% wstawki pdf
%\usepackage{pdfpages}
%% wstawki z kodem 
%\usepackage{listings}
%\usepackage{color}
\usepackage{url}
\makeatletter
\g@addto@macro{\UrlBreaks}{\UrlOrds}
\expandafter\def\expandafter\UrlBreaks\expandafter{\UrlBreaks%  save the current one
  \do\a\do\b\do\c\do\d\do\e\do\f\do\g\do\h\do\i\do\j%
  \do\k\do\l\do\m\do\n\do\o\do\p\do\q\do\r\do\s\do\t%
  \do\u\do\v\do\w\do\x\do\y\do\z\do\A\do\B\do\C\do\D%
  \do\E\do\F\do\G\do\H\do\I\do\J\do\K\do\L\do\M\do\N%
  \do\O\do\P\do\Q\do\R\do\S\do\T\do\U\do\V\do\W\do\X%
  \do\Y\do\Z}
  
%\usepackage[paper=A4]{typearea}
%marginesy 
\usepackage[ bindingoffset = 0cm, hmargin = 2cm, vmargin = 2cm]{geometry} 
%interlinia 
\linespread{1} 

\widowpenalty=10000 % ostatni wierszrkapitu nie zostanie przeniesiony na następną stronę 
\clubpenalty=10000 % pierwszy wiersz akapitu nie będzie kończył strony (nie używam tego ustawienia)
\hbadness= 1450 %% zmniejsza liczę wyświetlanych ostrzeżeń (można zwiększyć, ale bez przesady)
\hfuzz = 0pt %% tekst może sterczeć ma marginesie na 1,5pt (ok. 0,5mm)

\clubpenalty=10000 %nie pozostawia sierot
\brokenpenalty=10000 %nie dzieli stron je»eli podziaª wyrazu
\sloppy %zakaz wydªu»ania lini

%wcięcia 
\setlength{\parindent}{1cm} 
\setcounter{secnumdepth}{2} %only sections and subsections are numbered 
\setcounter{tocdepth}{2} %table of contents shows up to three levels 

%\newcommand\tab[1][1cm]{\hspace*{#1}}

\begin{document} 



%%%%%%%%%%%%%%%%%%%%%%%%%%%%%%%%%%%%%%%
%			STRONA TYTUŁOWA
%%%%%%%%%%%%%%%%%%%%%%%%%%%%%%%%%%%%%%%

\begin{titlepage} 
{\begingroup
\centering 

\vspace*{15\baselineskip} 

{\Huge Metody Odkrywania Wiedzy}
\vspace*{1\baselineskip}

{\huge Dokumentacja końcowa projektu}

\vspace*{3\baselineskip}
{\LARGE „Predykcja zużycia energii na podstawie danych czujnikowych”}
\\[\baselineskip]

\vspace*{20\baselineskip} 
{\Large 
Krzysztof Belewicz\\
Paweł Pieńczuk\par} 

\vspace*{1\baselineskip}
\today

\endgroup\clearpage}
\end{titlepage} 

%%%%%%%%%%%%%%%%%%%%%%%%%%%%%%%%%%%%%%%%%%%%%%%%%%%%%%%%%%%%%   
% Sprawko właściwe
%%%%%%%%%%%%%%%%%%%%%%%%%%%%%%%%%%%%%%%%%%%%%%%%%%%%%%%%%%%%%

%TODO nazwy chapterów ukradzione z elka.mine
%TODO ciężkie wzorowanie w tym co napisane/ będzie napisane
\large %trochę cheat na objętość

\begingroup
\let\clearpage\relax
\chapter{Opis projektu} %Interpretacja tematu

Celem projektu było wyznaczenie całkowitego zużycia energii dla zadanej chwili czasu, tzn. sumy poborów sprzętów AGD (kolumna ‘Appliances’) i oświetlenia (kolumna ‘Lights’). 
Zbiór danych został pozyskany z archiwum dostępnego na stronie: 
{\url{https://archive.ics.uci.edu/ml/datasets/Appliances+energy+prediction}}. Pojęciem docelowym jest wartość całkowitej pobieranej mocy przez gospodarstwo domowe.

TODO
%TODO jaki to rodzaj problemu+opis

\endgroup


\begingroup
\let\clearpage\relax
\chapter{Opis danych}

\section{Charakterystyka danych}

Dane wykorzystywane do eksperymentów zostały zebrane za pomocą sieci czujników w niewielkim domu w czasie 4.5 miesiąca. 
Składają się z:
\begin{itemize}
\item[$\bullet$] daty i godziny pomiaru,
\item[$\bullet$] poboru energii sprzętów domowych [$Wh$],
\item[$\bullet$] poboru energii oświetlenia [$Wh$],
\item[$\bullet$] pomiarów temperatury i wilgotności dla 8 różnych pomieszczeń ([\degree $C$], [$\%$]),
\item[$\bullet$] pomiarów temperatury i wilgotności dla zewnętrznej, północnej strony budynku ([\degree $C$], [$\%$]),
\item[$\bullet$] danych z pobliskiej stacji pogodowej:
	\begin{itemize}
	\item[$\circ$] temperatura powietrza [\degree $C$],
	\item[$\circ$] temperatura punktu rosy [\degree $C$],
	\item[$\circ$] ciśnienie atmosferyczne [$mm~Hg$],
	\item[$\circ$] wilgotność [$\%$],
	\item[$\circ$] prędkość wiatru [$m/s$],
	\item[$\circ$] widoczność [$km$].
	\end{itemize}
\end{itemize}

\section{Przygotowanie danych}
Każdy pomiar został uśredniony z 3 próbek wykonanych w równych odstępach co ok. 3,3 min. W ramach przygotowania danych, data i godzina pomiaru zostaną rozdzielone na dwie oddzielne kolumny -- ułatwi to późniejsze operacje na danych.
\endgroup


%\pagebreak
\clearpage

\begingroup
\let\clearpage\relax
\chapter{Opis algorytmów}

TODO\\
%TODO pakiety R i parametry ~notatka z poprzedniego sprawka
%TODO miara jakości, ocena modeli, procedura oceny

\endgroup



\begingroup
\let\clearpage\relax
\chapter{Selekcja atrybutów}
TODO\\
\endgroup


%\begin{figure}[H]%
%    \centering
%    \subfloat{{
%    	\includegraphics[width=.8\linewidth]{grafika/P1/LTSpice-amp.png}
%    }}%
%    %\caption{Wykres chara}%
%    \label{AC_Spice_Amp}%
%\end{figure}


\end{document}
